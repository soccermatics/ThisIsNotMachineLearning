
Over the previous two chapters we have built an approach to modelling which is open to changing its position in order to view systems in multiple ways. This approach encourages taking in to account many different points of view and discourages taking a single, universal perspective. Different useful views of a system are combined to get an overall understanding of the system.

This open-ended process of model building is necessarily value-laden endeavour.  We already saw an indication of this in our example of Hollywood film data: the same data could be used to predict blockbusters or to reflect over discrimination in the film industry. We also saw it when we discussed predictive policing (a subject to which we will return in this chapter). In the coming chapters we will see many more such examples of values. But first, in this chapter, we look in more depth at the relational approach and discuss how it can be used to help incorporate ethical thinking in to an open approach to modelling. 

\section{Afro-feminism} 

\label{sec:Afro-feminism}
\begin{displayquote}
``Knowledge without wisdom is adequate for the powerful, but wisdom is essential for the survival of the subordinate.'' Patricia Hill Collins \cite{collins2002black} \\
\end{displayquote}

We already introduced the relational approach in section \ref{sec:relational}. These include Ubuntu, ... enactive cognitive science and Afro-feminism. It is the latter of these which we look at more closely now, before (in the next section) using it to ground a relational ethics approach which aligns with the open modelling approach we looked at in the previous chapter.

It might, to some readers, appear unusual that we choose Afro-feminism around which to build an ethical approach to machine learning. After all, the writers who developed Black feminist thought were not initially concerned with the best way to use mathematical models. However, these writers were engaged in describing a way of thinking that resisted universalism. Moreover, they were concerned with describing the importance of practice and lived experience in the accumulation of knowledge. And they did so with an intimate knowledge of how inequality and discrimination can be perpetrated by systems when lived experience is neglected. Out of all of the relational approaches it is thus, Afro-feminism thought that offer us most when in providing concrete tools upon which we can build a relational approach. 

Syliva Tamale 


In this section, we detail the above points in more detail, with reference, in particular, to the work by two prominent advocates of Afro-feminist epistemology, Patricia Hill Collins and bell hooks. Then in the next section, we build these in to our own relational ethics approach to machine learning and mathematical modelling.

Drawing on core differences between the dominant Western tradition and the Afro-feminist perspective, Collins broadly identifies two types of knowledge about the world: \textit{book learning} and \textit{wisdom}. For Collins, book learning is reasoning about the world made from a distance in a rational way. This form of knowledge aspires to arrive at ``the objective truth'' that transcends context, time, specific and particular conditions. Wisdom, on the other hand, comes from concrete lived experience. It does not have an ambition to unify, but rather to document our experiences. Wisdom grows as we experience life, but is never complete.


ABEBA: I HAVE MADE CHANGES HERE TO ADDRESS DOUBLE USAGE OF KNOWLEDGE IN YOUR THESIS TEXT. OK?

Collins emphasizes that people are not passive cognizers that contemplate and grasp the world in abstract forms from a distance knowledge. Understanding instead emerges from concrete lived experiences \cite{collins2002black}. Formal education is thus not the only route to knowledge and in many cases is not the most important. Wisdom should be given high credence in assessing knowledge claims.  

For Collins, wisdom takes for granted that there exists an inherent connection between what one does and how one thinks. EXAMPLE. bell hooks \cite{hooks1989talking} (following a pedagogical approach proposed by Paulo Freire \cite{freire1996pedagogy}) emphasises taking a subject to subject approach, in contrast rather than as subject to object approach. When we undertake to study a social system, in particular, we should acknowledge our part of it: we should see ourselves as one subject addressing another subject. By resisting seeing the account of any one subject as ``definitive'' or ``authorative'', we ``create a climate where scholarship from diverse groups ... flourish[es] and [are] better able to appreciate the significance of scholarship that emerges from a particular race, sex, and class perspective.'' (page 45, \cite{hooks1989talking}). Our socio-cultural background --- often refered to as our identity --- necessarily plays a role in how we study another subject.  


The type of wisdom described by Collins is not worked out in isolation from others, but is developed in dialogue with the community. Patricia Hill Collins \cite{collins2002black} gives several examples in the context of African American culture which illustrate this. DISCUSSION page 261 and CHURCH page 263. ....



The distinction between book-learnt knowledge and wisdom experienced through life is especially important when the type of knowledge in question concerns \textit{oppression}, \textit{structural discrimination}, and \textit{racism}. At its heart, the Afro-feminist approach to knowing contends that concrete experiences are primary, and abstract reasoning secondary. It is wisdom, and not ``book learning'', that enables one to identify and resist oppression. It follows that concepts such as ethics and justice need to be grounded in concrete events informed by lived experience of the most marginalized, individuals and communities that pay the highest price when injustice is perpetrated. Knowing and being are active processes that are necessarily political and ethical. 

It is only that experience, which can change an unjust system or design a new approach to societal issues. Audre Lorde famously said that ``Those of us who stand outside the circle of this society's definition of acceptable women; those of us who have been forged in the crucibles of difference -- those of us who are poor, who are lesbians, who are Black, who are older -- know that survival is not an academic skill. It is learning how to take our differences and make them strengths. For the master's tools will never dismantle the master's house.'' \cite{lorde2003master} The experience of those who are oppressed is not merely useful in designing systems which can potentially lead to further oppression. It is primary.

\section{Ethics Built on Relationality}

\label{sec:ethics built on}

Let us summarise where we have come so far in this book. We started by acknowledging that complex systems are open-ended: they carry with them their history and a social context. They are ambiguous, messy, fluid, non-determinable, contextual and never completely understood. There are always new and different ways of looking at a complex system, and thus different ways of creating models. In some cases, most notably games or some physical systems, it is possible to close the system off. But there are many more systems, which remain open and for which we can never find a final all-encompasing view.

We then identified two broad approaches (or cultures) in modelling: one which is data-driven and emphasises prediction; the other which emphasises mechanisms and human understanding. For some closed systems, board games and even weather prediction, the data-driven approach often provides the best way of .

For open systems, both data-driven and mechanistic approaches can be employed, but it is here that the benefits of elucidating a mechanism are greatest. The mechanistic approach allows us to discuss and reason about models, compare our assumptions and reach conclusions in a way that cannot be achieved in a purely data-driven approach. The open approach engages us in an active discussion about the complex systems we want to model. As a result, we need to take our values and ethics in to our modelling approach; in the decisions we make about what to include and leave out of a model. In general, the more open a problem is, the more pressing ethical issues become.

It is here relational thinking comes in. We can think of the mathematical modelling itself as primarily book knowledge, a set of methods for building closed models of particular aspects of a system. On the other hand, the way in which we approach a system as a whole requires lived experience and wisdom: an understanding of the design decisions made when we use models and an understanding of the system we are applying our model to. A relational approach prioritises an open dialogue around modelling choices, over the closed knowledge provided by technical details of models.

bell hooks writes that "dialogue implies talk between two subjects, not the speech of subject and object.” \cite{hooks1989talking}. In terms of modelling, such a dialogue involves discussing and comparing the different ways we view a system, rather than silently applying one or a small number models to that system. A participatory discussion should be opened up about how to approach a problem. 

In the following subsections, we provide some principles which can help structure a relational approach. To make these ideas concrete, we consider the, in the context predictive policing: using algorithms to decide which areas of a city police should patrol. Such algorithms are used in many US cities 



 Here, we use predictive policing in order to illustrate the types of questions that relational ethics should ask. We don't necessarily give answers: although we encourage the reader to think of their own. In later chapters of the book, we will use these principles to challenge 

\begin{itemize}
    \item Paragraph the idea itself.  (Abeba can write long, David shorten)
    \item The literature (Abeba loads and loads)
    \item How to ask a question about predictive policing. (David)
\end{itemize}



\subsection{Subject or Object}
\label{sec:subjectobject}

The quote above from bell hooks makes a key distinction between subject and object. The 


In the predictive policing example, the people living in a city become objects to be monitored and controlled. 



\subsection{Disproportionally impacted}

\label{sec:disproportionally}
The harm, bias, and injustice that emerge from algorithmic systems varies and is dependent on the training and validation dataset, the underlying taken for granted assumptions of the model, and the specific context the system is deployed in, amongst other factors. However, one thing remains constant: individuals and communities that are at the margins of society are disproportionally impacted. Some examples include object detection \cite{wilson2019predictive}; search engine results \cite{noble2018algorithms}; recidivism \cite{angwin2016machine}; gender recognition \cite{buolamwini2018gender}; gender classification \cite{hamidi2018gender,barlassee2020}; and medicine \cite{obermeyer2019dissecting}. Wilson et.al.'s findings in \cite{wilson2019predictive}, for instance, demonstrate that object detection systems designed to predict pedestrians display higher error rates identifying dark skin pedestrians while light-skinned pedestrians are identified with higher precision. The use of such systems situates the recognition of subjectivity with skin tone where whiteness is taken as ideal mode of being. Furthermore, gender classification systems often operate under essentialist assumptions and operationalize gender in a trans-exclusive way resulting in disproportionate harm to trans people \cite{keyes2018misgendering,hamidi2018gender}.  


Given that harm is distributed disproportionately and that the most marginalized hold the epistemic privilege to recognize harm and injustice, relational ethics asks that for any solution that we seek, the starting point be the individuals and groups that are impacted the most. This means we seek to centre the needs and welfare of those that are disproportionally impacted and not solutions that benefit the majority. Most of the time this means not simply creating a fairness metric for an existing system but rather questioning what the system is doing, particularly examining its consequences on minoritized and vulnerable groups. This requires us to zoom out and draw the bigger picture. A shift from asking narrow questions such as \textit{how can we make a certain dataset representative?} to examining larger issues such as \textit{what is the product or tool being used for? Who benefits? Who is harmed? What are the factors that our model has taken into consideration (and what factors are left out as irrelevant). And are the factors we failed to consider or deemed irrelevant indeed so?} 

To some extent, the idea of \textit{centring the disproportionally impacted} shares some commonalities with aspects of \textit{participatory design}, where design is treated as a fundamentally participatory act \cite{slavin2016design} and even aspects human-centered design \cite{irani2010postcolonial} where individuals or groups whom technology is supposed to serve are placed at the centre. However, the idea of \textit{centring the disproportionally impacted} goes further than human-centered or participatory design as broadly construed. While the latter approaches can neglect those at the margins \cite{harrington2020forgotten}, shy away from power asymmetries and structural inequalities that permeate the social world, and ``mirror individualism and capitalism by catering to consumer's purchasing power at the expense of obscuring the hidden labor that is necessary for creating such system'' \cite{Lioyd2020} for the former, acknowledging these deeply ingrained structural hierarchies and hidden labour is a central starting point. In this regard, with a great emphasis on asymmetrical power relations, works such as Costanza-Chock \cite{costanza2018design}'s \textit{Design Justice} and Harrington \cite{harrington2020forgotten}'s \textit{The Forgotten Margins} are examples that provide insights into how centring the disproportionately impacted might be realized through design led by marginalized communities. 

%emphasize the importance of participation. Engaging with complex adaptive systems that surround us, Slavin \cite{slavin2016design} for example, stresses that ``every one of us (designers included) are nothing more than participants in''. 

The central implication of this in the context of a justice centred data practice is that minoritized populations that experience harm disproportionately hold the epistemic authority to recognize injustice and harm given their lived experience. Understandings these concepts and building just technologies therefore, needs to proceed from the experience and testimony of the disproportionately harmed. The starting point towards efforts such as ethical practice in machine learning or theories of ethics, fairness, or discrimination needs to centre the material condition and the concrete consequences an algorithmic tool is likely to bring on the historically marginalized. Having said that, these are efforts with extreme nuances and magnitudes of complexity in reality. For example, questions such as `how might a data worker engage vulnerable communities in ways that surface harms, when it is often the case that algorithmic harms may be secondary effects, invisible to designers and communities alike? What questions might be asked to help anticipate these harms?', `how do we make frictions, often the site of power struggles, visible?' are difficult questions but questions that need to be negotiated and reiterated by communities, data workers and model developers. 




\subsection{Modesty}

The first rule of thumb when dealing with complex systems is the realization and acknowledgement that the phenomena that we are modelling does not have one “right” single solution/answer. Any and all answers will always remain incomplete as complex systems are open and dynamic means that they never come to completion. Completion will simply mark the death of a system. This means that we should always be modest with our claims and models. 


A relational approach shifts toward a more humble and modest understanding complex systems such as people, knowledge and social systems. .... This is also a call for rethinking concepts such as data, ethics, models, matrices of oppression, and structural inequalities as inherently interlinked and processual.   

Such an approach cannot be thought of in terms of universal principles or a set of out-of-the-box tools that can be implemented. 


\subsection{Broaden understanding}

Not every human condition is not a problem to be solved. Complex systems, especially, within the domain of human and social affairs constitute long standing questions that have been contemplated for centuries with no clear answers means that 1) your model is not the first one to deal with these questions so go back and lean on the shoulders of giants and 2) these issues are unlikely to be “solved” once and for all. Thus, in approaching, framing, asking and “answering” questions of complex systems, look back at the historical body of work (outside your narrow and technical field) and build on that.       



\subsection{Critical reflection}

The search for irony is an internal process of challenging yourself as a researcher. One example is avoiding the temptation of low hanging fruit: if it feels that a theoretical result, which has no obvious application but is publishable and might attract interest from your peers, is within reach then make a brief note of how it might be solved (either for yourself or published in a blog) and then focus your attentions elsewhere. 
Use a variety of methods: Instead of polishing our own favourite lens and describing how it reflects white light, we should make use of every lens in our camera bag to get a multitude of different pictures and angles on our subject. We should treat the economy or social systems just as we treat the human body in Blanchard et al.’s example: as a portrait, as a mannequin, or as a pig. There are so many different ways to see ourselves and our society, we need to use them all to get as full a picture as possible. Examples can be seen in the study of collective animal behaviour, where different modelling and experimental approaches interact in the study of animal groups (Sumpter, 2010). 




\subsection{Reject notions of neutrality}

Systems cannot be modelled from the view from nowhere. The observer (and her objectives, values, and interests) is necessarily part of the model. When modelling human behaviour and social systems, which are contested and value-laden, there is no neutral lens. The more diverse the perspective, the fuller the picture the model produces.   


\subsection{Create ambitious theory}

????? Qualify for radicallity?

Develop entirely new lenses – radically theories of how to view complex systems. Some pointers in the direction include research in artificial life, using and developing online games where humans interact with simulations and investigating novel cellular automata (see figure 2). The common theme is an open-ended attempt to identify emergent phenomena, without ever trying to close the system with an exhaustive mathematical analysis. Instead of stifling the use of mathematics, a true complexity science pushes us to be more creative, to take risks and allow ourselves to be spectacularly wrong.  


\subsection{Embrace ambiguity}

Complexity science’s gradual slip back into reductive science is driven in part by the desire for control, certainty, universality, and absolute objectivity (again, the view from nowhere) (Birhane, 2021). However, when dealing with the messy, ambiguous and value-laden nature of complex systems, these desires are an illusion. In an academic ecology that rewards a false sense certainty, it is difficult to sell incomplete answers. Acknowledging that we never have a full grasp of complex phenomena or complete description of a system in its entirety — embracing ambiguity, being humble about our models, and being comfortable with “we don’t know” — can create a better culture for everyone.  



\subsection{Acknowledge power dynamics}

Since knowing is a relational affair, who enters into the knower-known relations matters. Within the fields of computing and data sciences, the \textit{knower} is heavily dominated by privileged groups of mainly elite, Western, cis-gendered, and able-bodied white men \cite{broussard2018artificial}. Given that knower and known are closely tied, this means that most of the knowledge that such fields produce is reduced to the perspective, interest, and concerns of such dominant group. Subsequently, not only are the most privileged among us restricted to producing partial knowledge that fits a limited worldview (while such knowledge, tools, models, and technologies they produce are forced onto all groups, often disproportionately onto marginalized people), they are also poorly equipped to recognize injustice and oppression \cite{Berenstain2016}. D'Ignazio and F. Klein \cite{d2020data} call this phenomenon \textit{the privilege hazard}. This means that minoritized populations 1) experience harm disproportionally and 2) are better suited to recognize harm due to their epistemic privilege \cite{on1993marginality} while the reverse holds for those building and deploying models.  

%In approaching knowing as a relational and active practice, we understand it as active, contextual, and placing human relations as central to knowledge and 

\subsection{The Master's tools}


\subsection{Checklists can be reductive}

Very nature of these systems, the list is inexausable. These systems are changing and moving. The number of items you can list is endless. 


\section{Relational versus Rational}

Rational presented as a given. 


Current data practices, for the most part, follow the rational model of thinking where data are assumed to represent the world ``out there'' in a ``neutral'' way. Yet, not only is it fallacious to assume complex social reality can be fully represented by data, the process of data collection, analysis and interpretation of results is a value-laden endeavour. In the process of data collection, for example, the data scientist decides what is worth measuring (making some things visible and others invisible by default) and how. In the process of data cleaning, rich information that provides context about which data are collected and how datasets are structured is stripped away. Emphasizing the importance of contexts for datasets, Loukissas \cite{loukissas2019all} has proposed a shift into thinking in terms of \textit{data settings} instead of \textit{datasets}. 

The rational worldview that aspires to an ``objective'' knowledge from a ``God’s eye view'' has resulted in the treatment of the researcher as invisible, their interests, values, and background as inconsequential. In contrast, for Afro-feminist thought, the researcher is an important participant in the knowledge production process \cite{nnaemeka2004nego}. For Sarojini Nadar \cite{nadar2014stories}, coming to know is an active and participatory endeavour with the power to transform. Consequently, data and our models portray and represent certain mode of reality while leaving out others.   
%narrative research, since it puts story telling at the centre, invites us to consider stories as `data with soul' \cite{nadar2014stories}. 


The relational approach can be contrasted to many Western approaches, which were concerned with finding ways of seperating these three aspects. For example, Descarte ....

Similarly, in the 1920's logical positivism sought to categorise statements we make about the world as being synthetic (about the complex real world), analytical (about the properties of models) and nonsense (about value judgements). 

\subsection{No boundaries}


Before we delve into that, it is worth reemphasising that while the rational worldview tends to see knowledge, people, and reality in general as stable, for relational perspectives, we are fluid, active, and continually becoming. Nonetheless, the relational vs rational divide is not something that can be clearly demarcated but overlaps with fuzzy boundaries. Some approaches might prove difficult to fit in either category while others serve to bridge the gap -- Harding's \cite{harding1992rethinking} \textit{Strong Objectivity} is one such example that links relational and rational approaches. Furthermore, the relational and rational traditions exist in tension with a continual push and pull. For example, complexity science is a school of thought that emerged from this tension.   


\subsection{Language games}

Western philosophy has investigated similar ideas to those contained within Ubuntu. In Anglo-American philosophy, these start with the later work of Ludvig Wittgenstein. He formaulated the idea of langauge games, which very much parallels the approach we take above of taking snapshots of a system.

Postmodernismis sometimes used in terms of derision by some scientists, who portray it as the opposite of a scientific approach. For example,

These criticisms of 
As Cilliers pointed out, in his book Complexity and Postmodernism, 

