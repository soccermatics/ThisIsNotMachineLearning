
Being open to different views of a complex system.


There is no definitive methodology for modelling of complex systems, just a set of plural practices. 

In this chapter we 
focus on several application areas.



\section{Animal behaviour}

The multiple views approach is also adopted when, for example, ant pheromone trails are modelled in terms of cycles of ant activity, formation and topology of the spatial patterns of trail networks, evolution of co-operation and chemical properties of the trails~\cite{sumpter2010collective}. Further examples are found in modelling the growth of tumours, genetic networks and ecological systems. Multiple views are also a prerequisite for modelling (more complex) human social systems~\cite{helbing2010pluralistic}. In adopting an Open ML approach, we simultaneously engage many different frameworks and views of a system, each designed to answer a different sub-question. We take different snapshots of the system and then use each of them to construct a bigger picture of the system. The more snapshots we include, the more complete the bigger picture. ML might help find the sharpest focus of one particular snapshot, but it can not tell us what is a good, overall picture.



\section{Football}

Team sports are more complex compared to board 
games, for example. They involve social, physical, tactical, and mental aspects. Team sports are however less complex than other 
systems such as human societies, 
financial systems, or human brains. 
Modelling the game of football, 
thus allows us to understand some of the challenges involved in modelling open systems, while still dealing with an application of (somewhat) limited scope.

\begin{figure}[t]
\includegraphics[width=11.5cm]{Source/images/Illustration_old.png}
\centering
\caption{Under the Open ML approach, the game of football can be modelled in many different ways. Here we illustrate: the prediction view (top right) simulates the game as a Poisson process; the pitch view uses models to evaluate impact of actions (bottom right); the society view uses the game to understand society at large (bottom left) (figure from \cite{Gregory2021Pace}); and the bio-mechanics view studies physical processes (top left). \label{football}} 

\end{figure}

A widely used model for predicting the outcome of a football match is Poisson regression~\cite{dixon1997modelling}. The central idea is that goals in the match 
are independent, occurring at a rate which depends on the relative quality of the teams and which can be estimated using regression methods.
This model is used by professional gamblers and bookmakers, since it outperforms betting strategies of the customers of the bookmakers (see e.g.~\cite{spann2009sports}). It is possible to include more factors, including events during the match, for example, in a neural network to improve predictions, giving a \textit{prediction} view of the game.

The prediction view is of little use to the players, who will have some sense of the strength of their opponents, and thus whether or not their team is likely to win, but can't be helped by a model (ML or otherwise) which sets probabilities to the outcome. Those playing the game want to understand specific details of their opponents' and their teammates' play which they can exploit during the match. Models that provide these insights can be found, with help of ML, through concepts such as pass probability and pass values, which (using historical data) evaluate the quality of actions~\cite{fernandez2019decomposing,sumpter2016soccermatics}. 

There are many other levels and dimensions to football, as Figure~\ref{football} shows. For example, the \textit{bio-mechanics} view looks at the body kinematics of players~\cite{ibrahim2019kinematic}. One example of the \textit{societal} view is statistical analysis of refereeing to reveal discrimination 
in decisions made~\cite{gallo2013punishing}. Another is the use of computer vision to investigate how sports commentators use words, such as `pace' and `power', when describing players with non-white backgrounds while words such as `hard work', 'effort' and 'mental skill' are used to describe white players. For example,~\cite{Gregory2021Pace} looked at how commentators described events on the pitch when they could and couldn't identify the ethnicity of the players.

Closed and Partially Open ML models can be, and are used in the approach outlined above, in the sense that regression, neural networks, and other methods are used to fit data. But their usage is secondary to finding different views of the sport, taken from different perspectives. Finding a view is sometimes referred to in ML as feature selection. But this terminology places the ML model as primary and the features as secondary. The problem with framing this process as feature selection is that it gives the model itself an aura of neutrality to which subjectively chosen features are added. In fact, the open-ended process of model building is always a necessarily value-laden endeavour. The Open ML approach, which we emphasize, places the ML model as a tool for fitting data, once we have found the view we are interested in. Open ML, then, is about finding a useful view for a certain problem, and combining the views to get an overall understanding of the system. The usefulness of the view subsequently cannot be entirely divorced from the modeller's objectives, motivations, and perspectives.  

\section{Human social behaviour}


